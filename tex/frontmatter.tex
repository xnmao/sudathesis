%---------------------------------------------------------------------------%
%->> Frontmatter
%---------------------------------------------------------------------------%
%-
%-> 中文摘要
%-
\intobmk\chapter*{中文摘要}\chaptermark{中文摘要}% 显示在书签但不显示在目录
\setcounter{page}{1}% 开始页
\pagenumbering{Roman}% 页码符号

论文第一页为中文摘要,主要是对论文的研究成果进行高度概括,力求用最简练的语言对论文的背景、研究目的、研究方法、主要研究工作与成果、创新点及结论进行高度的凝练。
硕士学位论文中文摘要约500字左右,博士学位论文中文摘要约1000字左右。

关键词一般为3--5个,每个关键词中间用分号分隔。

\keywords{关键词1;关键词2;关键词3;关键词4;关键词5}

\vspace*{\baselineskip}\hfill{}{\begin{tabular}{rcc}% 创建表格
	\makebox[5.5em][s]{作\hspace{\fill}者:} & 毛心楠 &\\
	\makebox[5.5em][s]{指\hspace{\fill}导\hspace{\fill}教\hspace{\fill}师:} & *** &
\end{tabular}}
%-
%-> 英文摘要
%-
\intobmk\chapter*{此处写上论文英文标题}\chaptermark{Abstract}% 显示在书签但不显示在目录
%\fancyhead[LO,RE]{\footnotesize 英文标题第1行\\英文标题第2行}% 页眉显示英文标题
\centerline{\zihao{3}\bfseries\sffamily{Abstract}}
\vspace*{18pt}% Abstract标题段后间距
{
%\setlength{\headsep}{\dimexpr 17.5pt + 1em\relax}% 如果页眉的英文标题换行需要补偿间距

The abstract of a dissertation is a summary of the research works.
The abstract should be concise and to the point, and the main content in the abstract should consist of the description of the research works, results, novelty and conclusions.

A maximum of 5 keywords should be contained, and semi-colons should be included between two keywords.

\KEYWORDS{Keyword 1; Keyword 2; Keyword 3; Keyword 4; Keyword 5}

\vspace*{\baselineskip}\hfill{}\begin{tabular}{rcc}% 创建表格
		\makebox[13.5ex][s]{Written\hspace{\fill}by:} & Xinnan Mao &\\
		\makebox[13.5ex][s]{Supervised\hspace{\fill}by:} & *** &
\end{tabular}
%\newpage% 补偿间距作用到最后1页
%\newpage% 补偿间距抵消双页排版问题
}
%---------------------------------------------------------------------------%
