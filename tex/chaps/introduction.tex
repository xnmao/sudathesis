\chapter{绪论}

\section{引言}

根据热力学第二定律,任何等温等压封闭系统倾向降低吉布斯能。
在没有外力的影响下,任何反应混合物也是如此。
比方,对系统中焓的分析可以得到合乎反应混合物的热力学计算。
反应中焓的计算方式采用标准反应焓以及反应热加成性定律(盖斯定律/盖斯定律)。
以甲烷(\ch{CH4})在氧(\ch{O2})中的燃烧反应为例:
\begin{equation}
	\ch{CH4 + 2 O2 -> CO2 + 2 H2O}
\end{equation}
能量计算须打断反应左侧和右侧的所有键结取得能量数据,才能计算反应物和生成物的能量差。
以$\Delta H$表示能量差,$\Delta$(Delta)表示差异,$H$则为焓等于固定压力下的热传导能量。%
$\Delta H$的单位为千焦耳(\unit{\kJ})或千卡(kcal)。

\begin{figure}[!htbp]
	\centering
	\includegraphics[width=0.7\textwidth]{graphic/introduction/sustainable\_energy\_future.jpg}
	\bicaption{可持续能源的未来。基于电催化的可持续能源景观示意图。\cite{RN190}}{Sustainable energy future. Schematic of a sustainable energy and scape based on electrocatalysis.\cite{RN190}}
	\label{fgr:sustainable_energy_future}
\end{figure}% https://science.sciencemag.org/content/355/6321/eaad4998

若已熟悉\LaTeX{}的用法,后文的格式说明可直接略过,因对使用者而言无须特别关注。
若已熟悉\LaTeX{}的用法,后文的格式说明可直接略过,因对使用者而言无须特别关注。
若已熟悉\LaTeX{}的用法,后文的格式说明可直接略过,因对使用者而言无须特别关注。
若已熟悉\LaTeX{}的用法,后文的格式说明可直接略过,因对使用者而言无须特别关注。
若已熟悉\LaTeX{}的用法,后文的格式说明可直接略过,因对使用者而言无须特别关注。
