\chapter{正文格式详细要求}

\section{文字格式要求}

\subsection{标题}

标题采取一级、二级、三级标题的格式(不建议设置四级标题),具体格式为:

一级标题:中文采用黑体,英文和阿拉伯数字采用Times New Roman,小二,段前24磅,段后18磅,章序号与标题之间空一个汉字符号,居中,所有平级的标题都用相同的格式;

二级标题:中文采用黑体,英文和阿拉伯数字采用Times New Roman,四号,段前24磅,段后6磅,章序号与标题之间空一个汉字符号,靠左对齐,所有平级的标题都用相同的格式;

三级标题:中文采用黑体,英文和阿拉伯数字采用Times New Roman,小四,段前12磅,段后6磅,章序号与标题之间空一个汉字符号,靠左对齐,所有平级的标题都用相同的格式;

这些格式均在“样式”中设置完成,直接引用样式;正文采用宋体/Times New Roman,小四。

如果标题过长需要分行书写的话,则第一行段前空相应的磅数,段后空0磅,最后一行段前空0磅,段后空相应的磅数,其余行均设置为段前0磅,段后0磅。

\subsection{段落文字}

中文采用宋体,英文和阿拉伯数字采用Times New Roman,两端对齐,段落首行左缩进2个汉字符。
1.5倍行间距,段前、段后均设置为0。

\section{页眉与页脚}

页眉和页脚:页眉距边界2.0厘米,页脚距边界1.75厘米。

页眉:从摘要开始,每部分均需要添加页眉。页眉文字下方加一条横线,线宽为0.75磅,长度与页面宽度一致。
奇数页页眉横线左侧为论文题目,右侧为章节标题;偶数页页眉横线左侧为章节标题,右侧为论文题目。中文采用小五,宋体。英文采用小五,Times New Roman。

页尾:页尾一般添加页码;封面、独创性声明、授权使用声明均没有页码,摘要、目录采用罗马数字“\uppercase\expandafter{\romannumeral1}、\uppercase\expandafter{\romannumeral2}、\uppercase\expandafter{\romannumeral3}...”表示,从“第一章 引言”开始至论文结束,页码用阿拉伯数字“1,2,3…”表示。

页码置于页面下方居中排列,采用Times New Roman小五字体。
中英文摘要的页码用罗马数字(如\uppercase\expandafter{\romannumeral1}、\uppercase\expandafter{\romannumeral2}、\uppercase\expandafter{\romannumeral3}等)编号,目录部分不编页码。

\section{公式格式要求}

若论文中存在公式,范本如下:
公式范本:
\begin{equation}
    1~\unit{\ug\per\kg} = \num{1e-06}~\unit{\mg\per\g}
\end{equation}

按照章节进行编号,编号格式为:公式右边对齐,(1-1)、(1-2)、(1-3),序号与表达式之间不加任何连线;有公式的行,行间距为单倍行距,段前段后各空3磅。
数字与单位之间须有空格,变量需要斜体。
