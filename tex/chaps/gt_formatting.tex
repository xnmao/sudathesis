\chapter{正文图表详细格式与示例}

与公式类似,图片、表格均按照章节编号,用两格阿拉伯数字表示,前一数字为章节的序号,后一数字为本章节内图片、表格的顺序号,两格数字之间用“-”表示。
例如:图1-1、图2-3、表1-2等。
图表编号与后面的图注中间空一个汉字符。

\section{图片详细格式与示例}

\subsection{图片详细格式}

图片应该能够清晰地表达出具体内容,图片中的术语、符号、单位等应该与正文表述中一致。

图片直接插入正文,图片横向尺寸不大于17厘米,纵向尺寸根据图片内容进行调节。

图片插入文章中时。设置为:单倍行距,段前12磅,段后6磅,居中。

图注包括图序与图释,宋体,五号,单倍行距,段前6磅,段后12磅。若图注为多行文字,第一行设置为段前6磅,段后0磅,最后一行为段前0磅,段后12磅,其余行均为段前0磅,段后0磅,居中。

如一个图由2个或者更多个子图组成时,各子图分别以a、b、c…等作为图序,并须有每个子图对应的图释。

博士学位论文图例需包含英文图例。

\subsection{图片示例}

\begin{figure}[!htbp]
	\centering
	\includegraphics[width=\textwidth]{graphic/introduction/sustainable\_energy\_future.jpg}
	\bicaption{可持续能源的未来。基于电催化的可持续能源景观示意图。\cite{RN190}}{Sustainable energy future. Schematic of a sustainable energy and scape based on electrocatalysis.\cite{RN190}}
	\label{fgr:sustainable_energy_future}
\end{figure}% https://science.sciencemag.org/content/355/6321/eaad4998

\section{表格详细格式与示例}

\subsection{表格详细格式}

采用三线表格式编写,字体五号,行间距为固定值20磅,段前3磅,段后3磅,中英文分别为宋体和Times New Roman。

表序与表名,例如“表2-1 实验使用仪器”,位于表的上方,宋体,五号,居中,段前12磅,段后6磅,行距为单倍行距,表序与表名之间空一个汉字符。

当表达较大无法在一页表示完全时,可以“续表”的形式在第二页继续,格式同前,只需要在每页表需序之前加“续”字即可,例如“续表2-1 实验使用仪器”。

若在表下方著名缩写全名、资料来源等补充资料,则此部分为宋体五号、行间距固定值20磅,段前6磅,段后12磅。

\subsection{表格示例}

\begin{table}[!htbp]
\bicaption{气相物质的自由能校正值。}{Free energy correction values for the gaseous species.}
\label{tbl:IrO2_gas}
\centering
\footnotesize% fontsize
% \setlength{\tabcolsep}{15pt}% column separation
% \renewcommand{\arraystretch}{1.2}%row space
\begin{tabular}{cS[table-format=+.2]S[table-format=.2]S[table-format=+.2]S[table-format=+.2]}
\toprule
Species  & $ {E}_{\mathrm{DFT}}\ [\unit{eV}] $ & $ \mathrm{ZPE}\ [\unit{eV}] $ & $ {-}TS\ [\unit{eV}] $ & $ G\ [\unit{eV}] $ \\
\midrule
\ch{H2}  &  -6.77                              &  0.27                         &  -0.40                 &  -6.90             \\
\ch{H2O} &  -14.22                             &  0.57                         &  -0.67                 &  -14.32            \\
\bottomrule
\end{tabular}
\end{table}

